\section{Nature of the problem}
\label{sec:NatureOfTheProblem}
%Give a short description of the nature of the problem and the eventual numerical methods, you have used.
%"Non-computational" algebra
%Show that you can rewrite this equation as a linear set of equations of the form
The aim of the first part of the project is to solving Schr\"{o}inger’s equations for one electron in a harmonic oscillator potential with angular momentum $l=0$. 
The radial part of the Schr\"{o}dinger’s equation is considered which is as follows
\begin{align}
 \left[ -\frac{\hbar^2}{2 m}  \frac{1}{r^2} \frac{d}{dr} r^2
  \frac{d}{dr} + V(r) \right]R(r) 
       = E R(r).
     \label{eq:NatureOfTheProblem1}
\end{align}
In order to solve this equation numerically, it is rewritten after a series of transformation and substitution as
\begin{align}
	-\frac{d^2}{d\rho^2} u(\rho) + \rho^2u(\rho)  = \lambda u(\rho)
	\label{eq:NatureOfTheProblem2}
\end{align}
\matref{eq:NatureOfTheProblem2} is discretized by writing the second derivative of $u(\rho)$ as 
\begin{align}
	\frac{d^2}{d\rho^2} u(\rho) =\frac{u(\rho+h) -2u(\rho) +u(\rho-h)}{h^2} +O(h^2)
	\label{eq:NatureOfTheProblem3}
\end{align}   
In \matref{eq:NatureOfTheProblem3} $h$ is the step length, and $\rho_{max}$ and $\rho_{min}$ are the maximum and minimum values of the variable $\rho$, respectively. 
For a given number of steps $n$, the step length is given as
\begin{align}
	h=\frac{\rho_{{max}}-\rho_{{min}} }{n}
	\label{eq:NatureOfTheProblem4}
\end{align}
In order to solve equation \matref{eq:NatureOfTheProblem2}, it is transformed into a matrix eigenvalue problem 
\begin{align}
	\v{A} \v{u} = \lambda \v{u}
	\label{eq:NatureOfTheProblem5}
\end{align}
in which $\v{A}$ is a tridiagonal matrix of the form
\begin{align}
	\v{A} = 
	\left( \begin{array}{ccccccc} \frac{2}{h^2}+V_1 & -\frac{1}{h^2} & 0   & 0    & \dots  &0     & 0 \\
                                -\frac{1}{h^2} & \frac{2}{h^2}+V_2 & -\frac{1}{h^2} & 0    & \dots  &0     &0 \\
                                0   & -\frac{1}{h^2} & \frac{2}{h^2}+V_3 & -\frac{1}{h^2}  &0       &\dots & 0\\
                                \dots  & \dots & \dots & \dots  &\dots      &\dots & \dots\\
                                0   & \dots & \dots & \dots  &\dots       &\frac{2}{h^2}+V_{n-2} & -\frac{1}{h^2}\\
                                0   & \dots & \dots & \dots  &\dots       &-\frac{1}{h^2} & \frac{2}{h^2}+V_{n-1}
             \end{array} \right) 
	\label{eq:NatureOfTheProblem6}
\end{align}
$\v{A}$ is obtained from \matref{eq:NatureOfTheProblem2}, with the approximation of the derivative of $u(\rho)$ given in \matref{eq:NatureOfTheProblem3} when omitting all later terms, by discretizing $\rho$ by
\begin{align}
	\rho_i= \rho_{{min}} + ih \hspace{1cm} i=0,1,2,\dots , n
	\label{eq:NatureOfTheProblem7}
\end{align}
This leads to the following Schr\"{o}dinger equation:
\begin{align}
	-\frac{u(\rho_i+h) -2u(\rho_i) +u(\rho_i-h)}{h^2}+\rho_i^2u(\rho_i)  = \lambda u(\rho_i)
	\label{eq:NatureOfTheProblem8}
\end{align}
which can be rewritten as
\begin{align}
	-\frac{u_{i+1} -2u_i +u_{i-1}}{h^2}+\rho_i^2u_i=-\frac{u_{i+1} -2u_i +u_{i-1} }{h^2}+V_iu_i  = \lambda u_i
	\label{eq:NatureOfTheProblem9}
\end{align}
in which $V_i = \rho_i^2$ is the harmonic oscillator potential.
When comparing this relation with the general eigenvalue problem in \matref{eq:NatureOfTheProblem5}, it is evident that the diagonal elements of the matrix $\v{A}$ is given by
\begin{align}
	d_i=\frac{2}{h^2}+V_i
	\label{eq:NatureOfTheProblem10}
\end{align} 
while all off diagonal elements are zero apart from those neighbouring the diagonal, which are all constants with the value
\begin{align}
	e_i=-\frac{1}{h^2}
	\label{eq:NatureOfTheProblem11}
\end{align}
This is exactly what is given in \matref{eq:NatureOfTheProblem6}. 

