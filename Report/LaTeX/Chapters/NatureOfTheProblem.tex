\section{Nature of the problem}
\label{sec:NatureOfTheProblem}
%Give a short description of the nature of the problem and the eventual numerical methods, you have used.
%"Non-computational" algebra
%Show that you can rewrite this equation as a linear set of equations of the form
First part of the project is aimed at solving Schrodinger’s equations for one electron. The radial part of the Schrodinger’s equation is considered which is as follows
\begin{align}
-\frac{\hbar^2}{2 m} \left ( \frac{1}{r^2} \frac{d}{dr} r^2
  \frac{d}{dr} - \frac{l (l + 1)}{r^2} \right )R(r) 
     + V(r) R(r) = E R(r).
     \label{eq:NatureOfTheProblem1}
\end{align}
In order to solve this equation numerically, it is rewritten after a series of transformation and substitution as
\begin{align}
	-\frac{d^2}{d\rho^2} u(\rho) + \rho^2u(\rho)  = \lambda u(\rho)
	\label{eq:NatureOfTheProblem2}
\end{align}
\fxnote{eq-ref} is discretized by assuming the second derivative of u
\begin{align}
	u''=\frac{u(\rho+h) -2u(\rho) +u(\rho-h)}{h^2} +O(h^2)
	\label{eq:NatureOfTheProblem3}
\end{align}   
In \fxnote{eq-ref} h is the step length, $\rho_{max}$ and $\rho_{min}$ are the maximum and minimum values of the variable ρ. For a given number of n step , h is given by
\begin{align}
	h=\frac{\rho_{\mathrm{max}}-\rho_{\mathrm{min}} }{n_{\mathrm{step}}}
	\label{eq:NatureOfTheProblem4}
\end{align}
Inorder to solve equation (3) it is transformed into a matrix eigenvalue problem 

CONTINUE.....